System Integrator - Nathan

\begin{itemize}
\item Describe the software development environment used to create the compiler.

The build environment of the compiler used a combination of OCaml,
Scala, and shell scripts to tie everything together.

We use OCaml for the source-to-source translator, compiling DotPar
code down to scala code.

We leverage the existing \texttt{scalac} compiler to compile the
results of the previous step, compiling the generated scala file down
to class files.

Then, we use a shell script to tie the translator and \texttt{scalac}
compiler together, and package the entire thing into a single
executable jar appropriate for execution directly on the JVM.

\item Show the makefile used to create and test the compiler during development.

<<<Copy-pasta>>>

\item Describe the run-time environment for the compiler.

Overall, the run-time environment of the generated programs is the
JVM, which executes the bytecode from the generated class files
packaged up into the final jar file.

At the next level down, the jar file also comes equipped with the
included Scala supporting class files: without these, the program will
not run as compiled by \texttt{scalac}. 

Finally, we include some supporting scala code that the generated
scala code relies on, which extends the behavior of the scala runtime
to provide the semantics required by the DotPar code.

\end{itemize}


Individual lessons learned:
 - Motivating people is hard.
 - Projects will grow to fill the time allotted to them.


Group lessons
 - Bleed now or bleed tomorrow.
