Language Guru - Sid

\section{Language Changes}
The LRM changed very little during the course of our implementation. This is
because we were cautious when writing the LRM and anticipated that features we
thought would be simple might end up being difficult to implement. In fact, our
LRM did not require any parallelism in a DotPar compiler implementation. This is
because the language could conceivably be run in an environment in which
parallelism is infeasible or undesirable; adding unnecessary restrictions on the
compiler writer seemed imprudent.

The biggest issue we we came across was dealing with overloading. For instance,
we wanted to overload the \verb$map$ function, but support neither function
overloading nor inheritance. Thus, we could not support using \verb$map$ on an
array of numbers and an array of strings, for example. To solve this, we
introduced an \verb$Any$ type which can be used for built-in functions. This
maintains the paradigms of our type system, but means that it is possible to get
a type error after semantic analysis, which violates some definitions of static
typing.

Parallelization was another challenging feature. But because it is the selling
point of the language we will discuss implementation here even though the LRM
does not strictly require it. We decided that our focus would be data
parallelism on \verb$for$ loops and special constructs like the \verb$map$ and
\verb$reduce$ functions and list comprehensions. We did not want to parallelize
lines outside of blocks. This would require constructing a full data flow graph
and generating code that would make wait for all dependencies to be computed
before executing a piece of code. While not infeasible, this introduced
additional complexity for minimal gain because the most expensive computations
tend to be in loops.

Due to time constraints, our analysis of data dependencies is limited to the
variable level. Thus, if each iteration of a loop from \verb$i$ to \verb$n$
writes to \verb$a[i]$ for some array \verb$a$, our code will think that there is
a dependence with between each iteration of these loops. This is known as a
\emph{false dependency}. Although every correct dependency analysis program
provably must generates some false dependencies, many common cases can be
optimized. There are well-known tests for array dependencies such as the
Banerjee GCD test and the Delta test. Because these tests are well-known and our
language provides ways to naturally parallelize maps and reduces even without
this analysis we decided not to implement these tests.

Another limitation of our parallelism is that we can only detect associativity
and commutativity of functions in special cases. Detecting these properties in
general cases is an open problem in the field. Without detecting these
properties, however, it becomes impossible to parallelize certain classes of
problems. For instance, to parallelize a \verb$reduce$, the reduction function
must be associative.

Another feature on our wishlist that we wanted but were forced to cut was Java
interoperability. Java interoperability would make our language practical in the
real world. Unfortunately, it would require generating an AST from bytecode in
order to do dependency analysis and to do type checking. That said, compiled
DotPar code can be used with Java code, just not the other way around.

\section{Language Implementation Tools}
To implement the language, we wrote the compiler in OCaml and used Scala as our
target language.

We used OCaml because we wanted to learn a functional paradigm. OCaml is also
often used for writing compilers, so many relevant tools and resources are
available. We used \verb$ocamllex$ and \verb$ocamlyacc$ for lexing and parsing,
respectively. The tools are similar to \verb$lex$ and \verb$yacc$, but have
slightly different syntax that is more fitting for OCaml.

We compiled down to Scala for several reasons. First, it gives us first-class
functions and closures for free. If we compiled down to something like Java,
implementing closures would involve creating wrapper classes for functions to
make them first-class and wrapping scopes in objects to implement closures.
Second, it was more natural to compile our imperative constructs into Scala than
it would have been to compile it to a functional language like Haskell or even a
multiparadigm language like OCaml, which errs towards the functional.
Additionally, Scala is threaded, and allows us to have more granular control
over parallelism than a language like JavaScript. Furthermore, Scala is
statically and strongly typed. Since our language does some type checking and
requires types to be specified, it makes sense to use a language that makes use
of these constructs since statically typed language tend to be faster than
dynamically typed languages. Finally, Scala runs on the JVM\@. While this was not
as essential as the other requirements, it is nice that any code that runs on
the JVM will be able to use compiled DotPar code. In the future, we could
leverage the virtual machine to use other code that runs on the JVM in DotPar
code.

We require no unusual libraries. An OCaml compiler, \verb$ocamlyacc$,
\verb$ocamllex$, and Scala 2.9 or greater must be installed.

\section{Language Implementation Practices}
Keeping LRM and compiler consistent:
\begin{itemize}
	\item{Iterative development}
	\item{Compilation to Scala worked first, then we added semantic analysis}
	\item{GitHub issues}
\end{itemize}
