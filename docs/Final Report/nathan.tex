System Integrator - Nathan

\section{Development Environment}

In order to effectively develop the Dotpar compiler, we used standard
development tools to streamline our development cycle, ensure a
consistent build across platforms, and generate binaries from Dotpar
code.

\subsection{Compiler Build Process}

We used a Makefile to manage our build and testing process, relying on
this standard tool in the open source world to abstract away the act
of building the compiler anew.

Holding with tradition, the Makefile provides familiar commands such
as \texttt{all}, \texttt{clean}, and \texttt{test}, which each in turn
generate a binary, clean up generated intermediate files, and run the
test suites on the generated binary, respectively. 

Additionally, we provide a finer-grained access to our test suites
with such commands as \texttt{parser\_test}, \texttt{semantic\_test},
and \texttt{full\_test} which run the seperate test suites that target
the major parts of our compiler in turn. The test suites themselves
are detailed in the ``Testing Methodology'' section.

To manage the code and facilitate collaboration, we used a source
control system known as git, which has recently become a popular tool,
especially in the open source world. In particular, we extensively
used branches in order to allow development in different areas at the
same time. Also, we leveraged the excellent git hosting tool
github.com, as detailed in the ``Github'' section.

The \texttt{dotpar} compiler itself is built on OCaml, as detailed in
``Language Implementation Tools''. We invoke the OCaml compiler from
our Makefile, compiling each separate part of the compiler with
\texttt{ocamlc -c} and then linking them together with \texttt{ocamlc
  -o} in discrete steps. The compiler itself has no dependencies on
external OCaml libraries that are not packaged with the standard OCaml
environment, which enhances it's portability.

\subsection{Makefile for the Project}

\begin{verbatim}
SRC = src
OCAML_PATH = $(SRC)/ocaml

all: bin compiler

bin:
	mkdir bin
	
compiler:
	cd $(OCAML_PATH); \
	ocamlc -c ast.ml; \
	ocamlc -c transform.ml; \
	ocamlyacc parser.mly; \
	ocamlc -c parser.mli; \
	ocamllex scanner.mll; \
	ocamlc -c scanner.ml; \
	ocamlc -c parser.ml; \
	ocamlc -c semantic.ml; \
	ocamlc -c parallelizer.ml; \
	ocamlc -c generate.ml; \
	ocamlc -c compile.ml; \
	ocamlc -c dotpar.ml; \
	ocamlc -o ../../bin/dotpar scanner.cmo ast.cmo transform.cmo \
			parser.cmo semantic.cmo parallelizer.cmo generate.cmo compile.cmo \
			dotpar.cmo

clean_ocaml:
	rm -f bin/dotpar
	cd $(OCAML_PATH); \
	rm -f *.cmo scanner.ml parser.ml parser.mli *.cmi

test: parser_test semantic_test full_test

parser_test: compiler
	python tests/parser_test.py
semantic_test: compiler
	python tests/semantic_test.py
full_test: compiler
	scala tests/full_test.scala

clean: clean_ocaml

.PHONY: all compiler test clean
\end{verbatim}

\subsection{Compilation Process}

The process of actually creating a binary from the generated compiler
is facilitated by a different program, a shell script named
\texttt{dotparc}. While \texttt{dotpar} takes Dotpar source code and
generates Scala code, \texttt{dotparc} reads in Dotpar files, runs
them through \texttt{dotpar}, and compiles the resulting Scala code to
an executable jar file. It does this by invoking the Scala compiler
\texttt{scalac} on the output Scala code, and packing the
\texttt{class} files into a jar along with the requisite Scala and
Dotpar support libraries. Then this jar is ready to be run on a JVM.

\subsection{Editor Plugins}

In order to facilitate writing tests, we wrote plugins for the popular
Vim and Emacs text editors to do code highlighting and indentation for
Dotpar files.

\subsection{Runtime Environment}

Our run-time is based on the JVM, as we leverage Scala, which compiles
down to JVM bytecode. More specifically, the generated JVM bytecode
relies on the included support Scala library files: without these, the
program will not run as compiled by \texttt{scalac}.

Finally, we include some supporting Scala code specifically tailored
to enabling our generated Dotpar code to run by extending the behavior
of the JVM and Scala runtimes to provide the semantics required by the
DotPar code.
