\section{Motivation}
Moore's Law, the trend that the number of transistors that can fit on
a chip doubles every 2 years, has held true for half a century,
driving innovations in all areas of technology. But recently chip
manufacturers have moved away from ever-increasing clock speeds, the
usual manifestation of Moore's law, and instead increase the number of
cores in CPUs to manage power dissipation.

As consumer and business demands for speed grow, programs must now be
written to take advantage of multiple cores in order to utilize the
power of CPUs. However, the imperative programming paradigm that has
prevailed in the programming community isn't conducive to safe,
parallel programming. Parallel programming is difficult and
synchronization bugs are notoriously difficult to fix.

Many solutions to this problem of parallelization have emerged, but
each has its drawbacks. We will briefly survey a few of the most
prominent existing solutions.

\subsection{Locking}
Locking controls access to memory accessible by several threads of
execution. However, this requires that the programmer manually lock
regions of code, which is tedious and error prone.  Historically, it
can lead to deadlock and race conditions, both of which are extremely
hard to debug.

\subsection{Task-based Concurrency and the Actor Model}
Task-based concurrency, used in languages like Ada and X10, doesn't
expose threads to the programmer. Simple computations, not threads are
the unit of parallelization. The actor model, present in SmallTalk and
Scala, treats different threads as actors that communicate with each
other via message passing. Since the actors don't share state, this is
much less error prone than locking since race conditions can be
avoided entirely. These models can be elegant, but they still require
manual division of labor on the part of the programmer, which
introduces errors and overhead, especially when refactoring code. We
seek to avoid imposing this burden on programmers.

\subsection{Functional Programming}
The functional paradigm minimizes side effects in favor of
deterministic functions. Under this model, functions could easily be
parallelized because they execute independently. Functional languages
like Haskell boast compilers that produce a great degree of
parallelization.

However, the functional paradigm is an unfamiliar way of coding for
many programmers and having state is often a preferable way of
modeling a program. In particular, object-oriented approaches have
proven useful in building large applications. These factors have
limited the adoption of purely functional languages.

Additionally, the benefits of parallelizing a purely functional
programming languages is limited because objects must be frequently
copied because they are immutable. This means that cache hit rates
drop significantly, main memory bandwidth becomes a bottleneck, and
garbage collection becomes a pain\footnote{http://bit.ly/dotpar1}.

\subsection{Implicit Parallelization}
Some efforts to write implicitly parallel, imperative languages have
been made\footnote{See http://bit.ly/dotpar2 and http://bit.ly/dotpar3
  for two examples.}. However, state- of-the-art compilers are not yet
capable of good implicit parallelization. While some cases of
potential parallelization are easy to recognize, many of them
difficult or impossible to find with static
analysis\footnote{http://bit.ly/dotpar4}.

With this in mind, implicit parallelization may sound daunting, but it
is important to remember that the compiler can miss many potential
opportunities for parallelization without sacrificing
performance. Even with a modest ability to find parallelization
opportunities, the performance of parallel programs is still limited
by the number of processors, not the number of threads that run
simultaneously. In fact, if too many threads are created, the overhead
of thread creation and execution will adversely affect performance.

% ------------------------------------------------------------
\section{Introducing DotPar}
With familiar C-like syntax, DotPar is a multi-paradigm language that
implicitly parallelizes code with minimal programming effort. By
abstracting away parallelization, DotPar enables programmers to focus
on designing and and architecting systems instead of on micromanaging
performance. Additionally, it provides the expressive power of
functional languages and includes some imperative paradigms derived
from C to ease the job of the programmer.

\subsection{Automatic Parallelization}
Based on static analysis of code and user annotations, DotPar can
detect a large class of parallelization opportunities, speeding up
execution considerably.

\subsection{Static and Strong Typing}
Keeping DotPar strongly and statically typed provides compile-time
check for common mistakes that should never make it to
production. This also enhances the compiler’s ability to parallelize
the code.

\subsection{First-class Functions and Lexical Closures}
By providing first-class functions and closures, DotPar provides a
great degree of expressiveness not possible in a language like
Java. However, it is not as restrictive as a purely functional
language like Haskell because it provides some imperative constructs
as well.

\subsection{Robustness and Security}
Through implicit parallelization, DotPar removes the risk of race
conditions and deadlock.  And, since it runs on the JVM, the code is
highly robust and portable. Furthermore, DotPar raises informative
errors that allow the user to debug code efficiently and quickly,
reducing development time.

\subsection{Memory Management and Garbage Collection}
Since DotPar runs on the JVM, garbage collection and memory management
are handled automatically, reducing programmer overhead.