\section{Group Lessons and Advice for Future Groups}
\subsection{Starting Early}
Starting early and spreading out work is an integral part of any project, especially one that requires as much work as this one. Having plenty of time at each stage is important to make sure that you are completely ready to move onto the next stage. While the design stage may seem less important than the implementation phases, being more thoughtful at that stage could have saved us many of the headaches we experienced down the road. Despite our best efforts, finishing still came down to late-night final pushes to fix the remaining bugs.

\section{Individual Lessons}
\subsection{Logan - Have a Backup Plan}
We started off the semester with a great group of individuals who are all capable and motivated. We were able to assign appropriate roles based on each member's skillset and what they wanted to do. Even with that, you have to count on things going wrong. I didn’t plan on getting very sick for over a month or having to miss a week and a half of school. The biggest problem was that it made it hard to communicate with the team and help coordinate who was taking over my responsibilities temporarily. Nathan really stepped up and helped keep the group on track with our schedule. I could have also been better about following up when people couldn’t make meetings due to other commitments. Occasionally, it would be the next weekly meeting before I got caught up on their progress. 

\subsection{Sid - Motivating People is Hard}
Each group member has other commitments that they need to balance with our project. Other classes, extracurricular activities and life often conflict with goals that we have set. There is no way to avoid the amount of time that needs to be invested, whether it put in today or tomorrow. To that extent motivating people to make this a priority and put in long hours during the middle of the semester can be very difficult and is something that we have struggled with. 

\subsection{Justin - Don't Be Afraid to Ask for Help}
During the course of the project I sometimes found myself struggling with a problem when it was actually unnecessary. This is because it came from a misunderstanding of a part of our design or not knowing everything about the new tools or concepts that we were using. Asking for help would have been a much more efficient way of trying to solve the problems that I was struggling with than just trying to keep ``googling'' everything that I could think of. Sometimes you don't want to ask because you feel like that is something you should already know, but supporting each other is an important part of having an effective group.

\subsection{Andrew - Pair Programming}
Pair programming is fun and helpful. When we worked in pairs on one computer we wrote better code, faster. Having two sets of eyes looking out for syntax errors and keeping logic consistent went a very long way in helping to reduce bugs and accomplish tasks in a more timely fashion. I felt this was helpful especially in the beginning when we were all learning ocaml because I would remember some syntax and functions and my partner would remember others. In this respect, we were able to learn from each other. The knowledge transfer experience contributed to the cohesion of our group.

\subsection{Nathan - Projects Will Take as Much Time as You Let Them}
I learned that no matter how ambitious or puny a project may seem at
first, the expectations and goals attached to that project can and
will grow and shrink as availability becomes more or less scarce: in
other words, projects will grow to fill the time allotted to them. We
originally envisioned the project being functionally done a week
before it was due, but since we had that buffer time, why not use it?
And now here we are, pushing this up to the last minute. This lesson
can also be known as ``self discipline is hard''.

\section{Advice for Future Groups}

\subsection{Make Group Deadlines Hard Deadlines}
In addition to the lessons we have learned, we would say the most important piece of advice to pass on is to stick to the deadlines you set for yourself. These create a domino effect and even a small pushback can lead you to run out of time. It is hard to think of a group deadline the same way that you think of a deadline set by a professor, but it is imperative for the completion of your project.

\section{Ideas for Improving the Course}
\subsection{Example Timeline}
Since this is the first time most people taking PLT have written a language, it would be useful to benefit from the experience of past groups and the professor by having a sample/suggested timeline for completing certain aspects of our language and compiler, not just the documents that we need to turn in. This can be fine-tuned with a TA based on the language specifics but it would be a good start and help groups be more realistic.

\subsection{Possible Deadlines for Parts of the Compiler}
While learning self-discipline is extremely important and a valuable lesson from the class, having to turn parts of your compiler into your TA would be a great way to make sure teams don't fall behind. There is nothing like an assigned deadline to make people get work done on time. These would have to be general to account for the differences in languages.

\subsection{More Specifics About Available Tools}
The inclass explainations of lex and yacc were very helpful for our group. We feel like this approach could be beneficial in helping groups understand the pros and cons of choosing ocaml versus java in addition to other tools that are available to them. This would also help to make the class portion of the course and the project part of the course more cohesive.
